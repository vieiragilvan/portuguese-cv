%% start of file `template.tex'.
%% Copyright 2006-2013 Xavier Danaux (xdanaux@gmail.com).
%
% This work may be distributed and/or modified under the
% conditions of the LaTeX Project Public License version 1.3c,
% available at http://www.latex-project.org/lppl/.


\documentclass[11pt,a4paper,sans]{moderncv}

\moderncvstyle{classic} 
\moderncvcolor{grey}    
\usepackage[utf8]{inputenc}

\usepackage[a4paper,margin=2cm]{geometry}
% personal data
\name{Gilvan}{Vieira}
\address{Engenheiro Humberto S. Camargo, 1137}{Campinas, São Paulo}{Brazil}
\phone[mobile]{+55~(19)~98317~7945}
\email{gilvandsv@gmail.com}

\begin{document}
\makecvtitle

\section{Education}
\cventry{2015--current}{PhD Candidate Computer Science}{University of Campinas (UNICAMP)}{}{}{}
\cventry{2013--2015}{MSc Candidate Computer Science}{University of Campinas (UNICAMP)}{}{}{}
\cventry{2008--2013}{BSc Computer Science}{Pontifical Catholic University of Goias}{}{}{}

\section{Master thesis}
\cvitem{title}{\emph{Acceleration of Finite Element NeoPZ library in NUMA Processors}}
\cvitem{supervisors}{Prof. Dr. Edson Borin and Phillipe R.B. Devloo}
\cvitem{description}{In NUMA architectures the naive distribution of data will hurt the performance. In this work we propose a template for the data migration and apply in the Finite Element library NeoPZ using a real civil engineering application.}

\section{Experience}
%\subsection{Vocational}
\cventry{2013--current}{Researcher}{Computational Mechanics Laboratory}{Campinas - SP, Brazil}{}{The Laboratory of Computational Mechanics is a permanent laboratory of the Department of Structures of the Faculty of Civil Engineering at UNICAMP who has as objective create applications for automate the solution of civil engineering problems.\newline{}%
Detailed achievements:%
\begin{itemize}%
\item I implemented in the Finite Element library NeoPZ a graph conversion method, where the entry is a mesh and the output a compressed sparse row graph to be used in the boost Cuthill–McKee implementation with a 2.4x speedup;
\item I execute in a daily basis the profiling and code optimization in the Finite Element Library NeoPZ, including changes in data structures, memory access patterns and including optimized libraries like Intel TBB and MKL;
\end{itemize}}

\cventry{2011--2012}{Internship}{DACT Eletrobras Furnas}{Goiania - GO, Brazil}{}{The internship consisted in create applications and tools to support the laboratory of concrete technology. The Eletrobras Furnas is a state company with expertise in the construction of hydroelectrics and energy supply.\newline{}%
Detailed achievements:%
\begin{itemize}%
\item I used data mining techniques for remove data noise and get consistency on physical and chemical data of high-performance concrete materials using Microsoft Excel;
\item I developed a file converter for two legacy simulation applications using C++ and QT to create the graphical interface;
\item I developed an application to adjust mechanical properties functions by non-linear regression for concrete using also C++ and QT.
\end{itemize}}

\section{Honors \& Awards}
\cventry{2014}{Facebook World Hackathon}{Participant}{Menlo Park, CA}{}{}
\cventry{2014}{Facebook Hackathon São Paulo}{First Place}{São Paulo, Brazil}{}{}
\cventry{2014}{Parallel Programming Challenge ERAD-SP}{First Place}{São Paulo, Brazil}{}{}
\cventry{2014}{Intel Student Expert UNICAMP}{Nominated}{São Paulo, Brazil}{}{}

\section{Languages}
\cvitemwithcomment{Portuguese}{Native}{Mother Tongue}
\cvitemwithcomment{English}{Fluent}{}

\section{Computer skills}
\cvdoubleitem{Languages}{C, C++, Java, Python}{}{}
\cvdoubleitem{Tools}{Intel VTune, Linux perf}{}{}
\cvdoubleitem{Libraries}{boost, Intel TBB, OpenMP}{}{}

\end{document}
