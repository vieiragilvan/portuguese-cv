%% start of file `template.tex'.
%% Copyright 2006-2013 Xavier Danaux (xdanaux@gmail.com).
%
% This work may be distributed and/or modified under the
% conditions of the LaTeX Project Public License version 1.3c,
% available at http://www.latex-project.org/lppl/.


\documentclass[11pt,a4paper,sans]{moderncv}

\moderncvstyle{classic} 
\moderncvcolor{grey}    
\usepackage[utf8]{inputenc}

\usepackage[a4paper,margin=2cm]{geometry}
% personal data
\name{Gilvan}{S. Vieira}
\address{Engenheiro Humberto S. Camargo, 1137}{Campinas, São Paulo}{Brazil}
\phone[mobile]{+55~(19)~98317~7945}
\email{gilvandsv@gmail.com}

\begin{document}
\makecvtitle

\section{Formação Acadêmica}
\cventry{2013--Março 2015}{Mestrado em Ciência da Computação}{Universidade Estadual de Campinas, UNICAMP}{\textbf{Título}: Aceleração da biblioteca de elementos finitos NeoPZ em processadores NUMA\newline{} \textbf{Orientadores}: Edson Borin (IC) e Philippe R. B. Devloo (FEC)}{}{}
\cventry{2008--2013}{Bacharel em Ciência da Computação}{Pontifícia Universidade Católica de Goiás, PUC GOIÁS}{\textbf{Título}: Roteamento de Veículos na Cadeia de Suprimentos \newline{} \textbf{Orientador}: Sibelius Lellis Vieira}{}{}

\section{Atuação Profissional}
%\subsection{Vocational}
\cventry{2013--current}{Pesquisador}{Laboratório de Mecânica Computacional, FEC UNICAMP}{Campinas - SP}{}{O LabMeC é um laboratório pertencente ao Departamento de Estruturas da Faculdade de Engenharia Civil, Arquitetura e Urbanismo (FEC) da Unicamp e visa o suporte a pesquisas na área de Mecânica Computacional, desenvolvendo programas de computador para automatizar o processo de resolução de problemas de engenharia.\newline{}%
\textbf{Detalhes}: Executo diariamente o perfilamento e otimização de código na biblioteca de elementos finitos NeoPZ, incluindo mudanças em estruturas de dados, padrões de acessos à memória e adição de bibliotecas como MKL, TBB e OpenMP.}

\cventry{2011--2012}{Estágio}{DACT Eletrobras Furnas}{Goiânia - GO}{}{Durante meu estágio no Departamento de Apoio e Controle Técnico (DACT) desenvolvi aplicações para apoiar a pesquisa em formulações físicas de concreto de alto desempenho.\newline{}%
\textbf{Detalhes}: Desenvoli aplicações utilizando C/C++ e QT com interfaces gráficas para conversão de arquivos entre aplicações legadas e para regressão não-linear de curvas de fluência.}

\section{Prêmios \& Títulos}
\cventry{2014}{Facebook World Hackathon}{Participante}{Menlo Park, CA, EUA \newline{}% 
Este evento reuniu todos os vencedores das competições ao redor do mundo e foi realizado na sede da empresa na Califórnia
}{}{}
\cventry{2014}{Facebook Hackathon São Paulo}{Primeiro Lugar}{São Paulo, Brasil \newline{}% 
O hackathon é um evento que reúne programadores, designers e outros profissionais ligados ao desenvolvimento de software para uma maratona de programação, cujo objetivo é desenvolver um protótipo em 24 horas
}{}{}
\cventry{2014}{Desafio de Programação Paralela ERAD-SP}{Primeiro Lugar}{São Paulo, Brasil \newline{}% 
O desafio consistiu na resolução paralela de um problema específico da área de computação em um ambiente com processadores Intel Xeon (16 cores) e um coprocessador Intel Xeon Phi (60 cores) utilizando OpenMP e/ou POSIX Threads}{}{}
%
\cventry{2014}{Intel Student Expert UNICAMP}{Nomeação}{São Paulo, Brasil \newline{}% 
O programa Intel Student Expert é uma oportunidade aos estudantes para atuarem como
“embaixadores” da Intel dentro dos seus respectivos campus universitários,
promovendo o Programa Acadêmico da Intel com as seguintes iniciativas}{}{}

\section{Idiomas}
\cvitemwithcomment{Inglês}{Profissional Avançado}{}

\section{Habilidades}
\cvdoubleitem{\emph{Linguagens}}{C, C++}{}{}
\cvdoubleitem{\emph{Ferramentas}}{Intel VTune, Linux perf, XCode}{}{}
\cvdoubleitem{\emph{Bibliotecas}}{boost, Intel TBB, OpenMP}{}{}

\end{document}
